% !TeX spellcheck = en_GB
% !TeX root = memoco-report.tex

\section{Introduction}
\label{chap:introduction}

\subsection{Description of the problem}
The task of the exercise was to develop two algorithms capable of solving the TSP problem, specialized to the domain of drilling holes in electric panels. The first algorithm implements an exact methods using the IBM CPLEX optimization suite. The second algorithm is an approximated heuristic, inspired to the Lin-Kernighan algorithm\cite{LinK73}.\\ Both algorithms have been tested on synthetic instances produced taking into account the applicative domain. The two algorithms are described in the following pages, and the performed tests and results are listed in the last section.

\subsection{Description of provided material}
%\minitoc
The delivered material includes all the produced code and everything necessary to run the program.
The provided archive comes with the following structure:
\renewcommand*\DTstylecomment{\rmfamily\color{blue}\textit}
\dirtree{%
	.1 /.
	.2 bin/\DTcomment{Binary file folder}.
	.2 build/\DTcomment{Folder for build file}.
	.2 files/.
	.2 instances/\DTcomment{Contains some test instances}.
	.2 plots/.
	.2 src/.
	.3 utils/\DTcomment{Utilities}.
	.4 cpxmacro.hpp.
	.4 params.hpp.
	.4 python\_adapter.hpp.
	.4 variadic\_table.hpp.
	.4 yaml\_parser.hpp.
	.3 calibrate.cpp\DTcomment{Main file to launch calibration}.
	.3 config.yml.
	.3 CPLEX.cpp.
	.3 CPLEX.hpp.
	.3 IteratedLK.cpp.
	.3 IteratedLK.hpp.
	.3 LK.cpp.
	.3 LK.hpp.
	.3 main.cpp\DTcomment{Program main}.
	.3 Pair.cpp.
	.3 Pair.hpp.
	.3 plot\_script.py.
	.3 Tour.cpp.
	.3 Tour.hpp.
	.3 TSPinstance.cpp.
	.3 TSPinstance.hpp.
	.3 TSPsolution.cpp.
	.3 TSPsolution.hpp.
}

\subsubsection{How to run}
The program has the following dependences:
\begin{itemize}
	\item \texttt{g++ 7} or higher;
	\item \texttt{IBM ILOG CPLEX Optimization Studio 12.8} or higher;
	\item \texttt{Python 2.7}.
\end{itemize}

To test the program, some instances should be generated or imported by placing them in the \texttt{instances/} folder. To generate some instances set \texttt{generate\_instances: true} in the \texttt{config.yml} file, and the other parameters as desired. If \texttt{N\_min = 20, N\_incr = 5, N\_max = 30} three \texttt{csv} files containing the coordinates of each point will be generated, with size 20, 25 and 30 respectively, and will be placed in the instances folder. After that it is possible to run the exact method or the heuristic on these instances. To do that add to the list in \texttt{instances\_to\_read} the filename (without extension) of the files that should be read and set \texttt{solve\_heur} and \texttt{solve\_cplex} as desired. Some reasonable parameters for the heuristic are already set, but it is possible to change them if necessary. The script can then be run from a Linux environment with \texttt{make \&\& ./bin/main}. The algorithm writes its output in file \texttt{solLK.txt} and \texttt{solCPLEX.txt} under the \texttt{files/} folder. These files are written between the various iterations, so you can check them to monitor the progress. Additionally running the heuristic also produces an image of the final solution and places it in folder \texttt{plots/}.\\


